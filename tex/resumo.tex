

O \emph{resumo} é um texto inaugural para quem quer conhecer o trabalho, deve conter
uma breve descrição de todo o trabalho (apenas um parágrafo). Portanto, só deve
ser escrito após o texto estar pronto. Não é uma coletânea de frases recortadas
do trabalho, mas uma apresentação concisa dos pontos relevantes, de modo que o
leitor tenha uma ideia completa do que lhe espera. Uma sugestão é que seja composto
por quatro pontos: 1) o que está sendo proposto, 2) qual o mérito da proposta, 3)
como a proposta foi avaliada/validada, 4) quais as possibilidades para trabalhos
futuros. É seguido de (geralmente) três palavras-chave que devem indicar claramente a que se
refere o seu trabalho. Por exemplo: \emph{Este trabalho apresenta informações úteis a produção de trabalhos
científicos para descrever e exemplificar como utililzar a classe \LaTeX\ do
Departamento de Ciência da Computação da Universidade de Brasília para gerar
documentos. A classe \unbcic\ define um padrão de formato para textos do CIC, facilitando a
geração de textos e permitindo que os autores foquem apenas no conteúdo. O formato
foi aprovado pelos professores do Departamento e utilizado para gerar este documento.
Melhorias futuras incluem manutenção contínua da classe e aprimoramento do texto
explicativo.}