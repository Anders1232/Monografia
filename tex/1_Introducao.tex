A tecnologia tem se tornado cada vez mais ubíqua na sociedade. Com o advento da Internet, a interação dos seres humanos com a tecnologia cresceu bastante, gerando um tráfego imenso de dados, e com isso a necessidade de processamento em larga escala. Nesse cenário, emergiu o conceito de nuvem computacional, um paradigma que permite processamento em larga escala sem ser necessário que o usuário tenha em mãos hardware com tamanha capacidade computacional.

Assim, grandes empresas da área de Tecnologia da Informação, como o \textit{Google} \cite{Google} e a \textit{Microsoft} \cite{Microsoft}, possuem vários \textit{datacenters} com uma imensa quantidade de computadores interligados via rede, os quais disponibilizam esses recursos de forma virtualizada a usuários que necessitem de processamento e de armazenamento em larga escala. A disponibilidade dessa capacidade de computação tem gerado uma revolução na forma como os serviços computacionais são disponibilizados na Internet.
Dessa forma, pequenas empresas agora conseguem prover serviços em larga escala sem necessitarem de um grande investimento em infraestrutura computacional, e grandes empresas conseguem reduzir custos com equipamentos \cite{WhatIsCloudComputing}.

Como existem vários provedores de nuvem e cada um deles tem seus pontos fortes e fracos, surgiu então a ideia de criar uma plataforma que utilize serviços de vários provedores de nuvem, podendo explorar o ponto forte de cada um deles. Assim, emergiu o conceito de Federação de Nuvens\cite{6427607}, que são plataformas nas quais os usuários conseguem o máximo de flexibilidade provido pela combinação de funcionalidades que os distintos provedores de nuvem disponibilizam a seus usuários.

Todavia, como toda nova tecnologia, ela possui seus próprios desafios, pois desenvolver uma plataforma com tamanha flexibilidade requer uma arquitetura muito bem projetada, implementada e capaz de ser eficiente, concomitantemente em que sua interface seja agradável ao usuário. Existem várias propostas para federações de nuvens, entra as quais é possível citar Demchenko\cite{6427607} e Buyya\cite{Buyya:2010:IUF:2143583.2143586}. Uma plataforma de Federação de Nuvens que tem sido continuamente evoluída é o BioNimbuZ \cite{BioNimbuZ_Breno_Deric} \cite{BioNimbuZ_Closer} \cite{BioNimbuZ_6846526} \cite{Saldanha2012} \cite{6732620_BioNimbuZ_ACOsched} \cite{BioNimbuZ_Willian_C99} \cite{closer12_BioNimbuZ_AHP} \cite{Saldanha_BioNimbus}, desenvolvido no \acrfull{LABID} da \acrfull{UnB} por alunos de graduação e pós-graduação.

O BioNimbuZ é uma plataforma para nuvens federadas para a execução de \textit{workflows}, inicialmente de Bioinformática, mas atualmente sua arquitetura suporta \textit{workflows} de propósito geral. Atualmente o BioNimbuZ conta com vários algoritmos de escalonamento, entretanto, nenhum deles trata do escalonamento para plataformas heterogêneas. Diante do exposto, este trabalho propõe um novo algoritmo de escalonamento para o BioNimbuZ, cujo objetivo é distribuir tarefas para \acrshort{CPU}s e \acrshort{GPU}s das máquinas virtuais providas pelos diversos provedores de nuvem.


\section{Objetivos}
Este trabalho tem como objetivo principal desenvolver um escalonador para a plataforma BioNimbuZ, que seja capaz de escalonar tarefas para arquiteturas heterogêneas, ou seja, compostas por \acrshort{CPU}s e \acrshort{GPU}s. Para cumprir este objetivo geral, os seguintes objetivos específicos devem ser alcançados: 

\begin{itemize}
	\item Analisar algoritmos de escalonamento para plataformas heterogêneas;
%	\item 
	\item Desenvolver um escalonador capaz de distribuir tarefas para plataformas heterogêneas;
	\item Integrar o escalonador desenvolvido ao BioNimbuZ;
	\item Analisar o desempenho do escalonador proposto.
\end{itemize}

\section{Estrutura do Trabalho}
Este trabalho contém, além deste capítulo introdutório, mais quatro capítulos. O Capítulo 2 discorre sobre computação em nuvem e federações de nuvens computacionais. O Capítulo 3 descreve a plataforma de nuvens Federadas BioNimbuZ. O Capítulo 4 aborda a atividade de escalonamento e propõe o algoritmo que será implementado. O Capítulo 5 discorre sobre como foi o processo de implementação do escalonador proposto. Por último, o Capítulo 6 apresenta as conclusões e alguns trabalhos futuros. O Apêndice detalha quais máquinas foram utilizadas nos testes que foram feitos.