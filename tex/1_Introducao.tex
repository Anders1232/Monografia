

\section{O problema do escalonamento}

O escalonamento é o método pelo qual trabalho, definido por algum conjunto de características, é atribuído à recursos que capazes de completá-lo. O Problema do Escalonamento, que é a busca do escalonamento no qual o tempo de excução do conjunto de trabalhos é mínimo, é NP-completo\cite{ULLMAN1975384}. De acordo com Sipser\cite{SIPSER}, um problema A é NP-completo se:

\begin{enumerate}
	\item A pertence à classe de problemas NP, e
	\item Todo problema B \begin{math}\in\end{math} NP é redutível em tempo polimonial  a A.
\end{enumerate}

Um problema pertence à NP se a for possível verificar a validade de uma solução em tempo polimonial.\cite{SIPSER}\cite{Cook:1971:CTP:800157.805047}.

Por mais que exista a dificuldade teórica supracitada, isso não impediu a evolução dos Sistemas Operacionais, os quais foram capazes de fazer escalonamento de vários processos mesmo na época de \acrfull{CPU}s tinham apenas um núcleo. Popularizando dessa forma os computadores pessoais na década de 80.

Atualmente, as \acrshort{CPU}s possuem vários núcleos, e são capazes de ter mais de um contexto carregado por núcleo, vide Ryzen™ Threadripper™\cite{Ryzen}. O que faz com que seja necessário que o processo de escalonamento leve em consideração como o mesmo será distribuído(ou não) entre os núcleos.

Além do escalonamento de processos para serem executados na \acrshort{CPU}, o problema do escalonamento também aparece em várias outras situações na computação:

\begin{itemize}
	\item Escalonamento de processos que farão swapping.
	\item Escalonamento de requisições que farão acesso ao disco.
	\item Escalonamento de pacotes que serão enviados pela rede.
	\item Escalonamento de tarefas para serem executadas em máquinas virtuais numa nuvem.
\end{itemize}

Esta monografia focará no último tópico citado. O qual recentemente presencio ua ascenção do uso de unidades de processamento gráfico(\acrfull{GPU}) para processamento de propósito geral (\acrfull{GPGPU}\cite{Dimitrov:2009:USA:1513895.1513907}\cite{Yang:2010:GCM:1809028.1806606}. 

\section{Unidade de Processamento gráfico}

As \acrshort{GPU}s surgiram inicialmente como hardware dedicado embutido para acelerar a renderização de imagens em jogos eletrônicos de arcades por volta dos anos 80\cite{1156160}.

Com o passar das décadas e a evolução da tecnologia, as \acrshort{GPU}s ficaram cada vez mais poderosas e complexas, principalmente para atender à rápida evolução dos jogos eletrônicos. Quando o computador pessoal popularizou, surgiu as \acrshort{GPU}s como as conhecemos, como placas acopláveis à placa mãe focadas em processamento gráfico.

Na última década, as unidades de processamento gráfico(\acrshort{GPU}s) ascenderam como opção para processamento paralelo. Em especial operações que envolvam matrizes conseguem ter um ganho de performance considerável\cite{Dalton:2015:OSM:2835205.2699470}, como processamento de imagens e vídeos\cite{Mulligan:2012:GSE:2168556.2168612}.

Atualmente provedores de nuvem estão provendo máquinas virtuais com \acrshort{GPU}s, como a \acrfull{AWS}\cite{AWS_GPU}, o \acrfull{GCP}\cite{Google_Cloud_GPU}, \textit{Microsoft Azure}\cite{Azure_GPU} e a \textit{IBM Cloud}\cite{IBM_Cloud_GPU}.

\section{Computação em Nuvem}

A Computação em Nuvem é um sistema distribuído que disponibiliza serviços de computação ao usuário, abstraindo informações sobre como esse serviço é provido. Existem muitas características que ajudam a definir a Computação em Nuvem, mas as consideradas principais são\cite{NIST_CLOUD_DEFINITION}:
\begin{itemize}
	\item \textit{On-demand self-service};
	\item Fácil acesso via rede;
	\item Recursos virtualizados.
	\item Alta escalabilidade, com capacidade de redimensionamento durante a execução; e,
	\item Serviço mensurado.
\end{itemize}

Computação em nuvem é considerado um novo paradigma para provisionamento de infraestrutura de computação, o qual transfere o local da infraestrutura para a rede. Porém parte dos conceitos sobre os quais esse conceito se baseia não são novos.\cite{Vaquero:2008:BCT:1496091.1496100_Cloud_definition}CITAR CITAÇÕES DO ARTIGO BASE.
Na Computação em Nuvem, existem dois atores principais: os usuários de um serviço que está na Nuvem e o provedor da Nuvem. Existem três tipos de serviços pricipais que são\cite{NIST_CLOUD_DEFINITION}:

\begin{itemize}
	\item \textbf{\textit{Infrastructure As A Service:}} O provedor disponibiliza recursos computacionais virtualizados. Capazes de serem reconfigurados dinamicamente;
	\item \textbf{\textit{Plataform As A Service:}} Uma plataforma sobre a qual os usuários podem implantar softwares ou serviços é disponibilizada. Os programas implanatado podem fazer uso de algumas funcionalidade fornecidas em forma de bibliotecas, linguagens de programação, entre outros;
	\item \textbf{\textit{Software As A Service:}} O usuário tem acesso a softwares que estão sendo executados na Nuvem Computacional. O que permite migrar requisito computacional do equipamento onde o serviço executa para a Nuvem.
\end{itemize}

Como cada provedor de nuvem possui seu próprio método de precificação, e uma interface própria para comunicação com os usuários de seus serviços, existe um movimento de federação de nuvens computacionais com o objetivo de minimizar dependência para com os provedores do serviço e também reduzir custos.

\section{Nuvens Federadas: BioNimbuZ}

O BioNimbuZ é uma plataforma livre de nuvens federadas para execução de workflows. Desenvolvido no laboratório de Bioinformática e Dados(LABID) por alunos de graduação e pós-graduação. Originalmente proposta por Saldanha\cite{Saldanha_BioNimbus} e refinada por alunos de iniciação científica, graduação, metrado e doutorado.\cite{closer12_BioNimbuZ_AHP}\cite{BioNimbuZ_6846526} \cite{6732620_BioNimbuZ_ACOsched}\cite{BioNimbuZ_Breno_Deric}\cite{BioNimbuZ_Vegara}\cite{BioNimbuZ_Willian_C99}

Implementado utilizando uma arquitetura de camadas, o BioNimbuZ possui 4 camadas, descritas a seguir:
\begin{itemize}
	\item Camada de Aplicação: Responsável por prover a interface de comunicação com o usuário, seja via uma interface gráfica(GUI), seja via \textit{web}. A partir dessa camada o usuário pode enviar \textit{workflows} para serem executados e fazer \textit{upload} do arquivos necessários. Além de poder acompanhar o andamento de seus \textit{workflows} e poder obter, caso queira, o resultados parciais que já tiverem sido produzidos. 
	
	\item Camada de Integração: Tem como objetivo de integrar as Camadas de Aplicação e de Núcleo, fazendo uso do \textit{framework} REST para prover de forma prática essa funcionalidade.
	
	\item Camada de Núcleo: Realiza toda a gerência da federação, incluindo: escalonamento de tarefas, controle de acesso de usuários, descobrimento de recursos e provedores, prover serviços de elasticidade, monitoramento, armazenamento entre outros.
	
	\item Camada de Infraestrutura: Disponibiliza uma interface de comunicação do BioNimbuZ com os provedores de nuvem. Utilizando \textit{plugins} para mapear requisições provenientes da Camada de Núcleo para comandos específicos de cada provedor.
\end{itemize}

Desenvolvido em Java, utiliza o Apache Zookeeper\cite{Zookeeper} para coordenação do sistema distribuído e o Apache Avro\cite{Avro} para serialização de dados. É capaz de utilizar os serviços de nuvem da \textit{Microsoft Azure}, \textit{\acrshort{AWS}} e a \textit{\acrshort{GCP}}.


\subsection{Software Livre}
O BioNimbuZ está disponível sob os termos da \acrfull{GPL}, a licença do Projeto \acrfull{GNU}. O Projeto GNU foi criado por Richard Stallman em 1983, com o objetivo de fazer um sistema operacional que respeitasse a liberdade dos usuários de utilizar os software que possuem da forma que lhe for mais conveniente. Essas liberdades são\cite{Free_Software}:

\begin{itemize}
	\item Liberade de rodar o programa da forma que quiser, e para qualquer propósito.
	\item Liberdade de estudar como o programa funciona, e poder modificar ele para o mesmo faça a computação da forma que o usuário quiser.
	\item Liberdade para redistribuir cópias.
	\item Liberdade de redistribuir cópias das versões modificadas para outros.
\end{itemize}

Para que essas liberdades sejam exercidas é necessário acesso ao código fonte. Atualmente o projeto GNU conseguiu colocar em produção o sistema operacional GNU. As distribuições mais populares do GNU utilizam o kernel Linux desenvolvido por Linux Towards, por mais que seja possível usar outros kernels como o Hurd\cite{Hurd} e o do FreeBSD\cite{Debian_kFreeBSD}.

Hoje em dia, softwares livre são robustos o bastante para opções válidas perante softwares proprietários. Principalmente em ambiente acadêmico, pois o compartilhamento de conhecimento é fundamental para o desenvolvimento da ciência e tecnologia.




\iffalse

%\url{http://www.escritacientifica.com/}

%%%%%%%%%%%%%%%%%%%%%%%%%%%%%%%%%%%%%%%%%%%%%%%%%%%%%%%%%%%%%%%%%%%%%%%%%%%%%%%%
%%%%%%%%%%%%%%%%%%%%%%%%%%%%%%%%%%%%%%%%%%%%%%%%%%%%%%%%%%%%%%%%%%%%%%%%%%%%%%%%
%%%%%%%%%%%%%%%%%%%%%%%%%%%%%%%%%%%%%%%%%%%%%%%%%%%%%%%%%%%%%%%%%%%%%%%%%%%%%%%%
\section{Trabalho de Conclusão de Curso}%
Todos os cursos do \acrfull{CIC} da \acrfull{UnB} exigem a produção de um
texto científico como requisito para formação. %(veja \refAnexo{NormasGerais} para mais detalhes)
As etapas desta monografia/dissertação/tese devem seguir o \emph{método científico}.

%%%%%%%%%%%%%%%%%%%%%%%%%%%%%%%%%%%%%%%%%%%%%%%%%%%%%%%%%%%%%%%%%%%%%%%%%%%%%%%%
%%%%%%%%%%%%%%%%%%%%%%%%%%%%%%%%%%%%%%%%%%%%%%%%%%%%%%%%%%%%%%%%%%%%%%%%%%%%%%%%
%%%%%%%%%%%%%%%%%%%%%%%%%%%%%%%%%%%%%%%%%%%%%%%%%%%%%%%%%%%%%%%%%%%%%%%%%%%%%%%%
\section{Metodologia Científica}%
Ciência (do Latim \emph{scientia}, traduzido como ``conhecimento'') é uma forma
sistemática de produzir conhecimento (via método científico), ou o nome dado a
estrutura organizada do conhecimento obtido.

O método científico é um conjunto de regras básicas de como proceder para produzir
conhecimento, criando algo novo ou corrigindo/incrementando conhecimentos
pré-existentes. Consiste em juntar evidências empíricas verificáveis baseadas na
observação sistemática e controlada, geralmente resultantes de experiências ou
pesquisa de campo, e analisá-las logicamente.

Esta ideia foi formalizada por Newton em sua obra \emph{Philosophiae Naturalis
Principia Mathematica}~\cite{newton1833philosophiae} da seguinte forma:
\begin{enumerate}
	\item Não se deve admitir causas das coisas naturais além daquelas
	que sejam verdadeiras e sejam suficientes para explicar seus fenômenos.
	\item Efeitos naturais do mesmo gênero devem ser atribuídos as mesmas causas.
	\item Características de corpos são consideradas universais.
	\item Proposições deduzidas da observação de fenômenos são
	consideradas corretas até que outro fenômeno mostre o contrário.
\end{enumerate}%

Uma abordagem para esta metodologia é seguir os seguintes passos:
\begin{description}
	\item[Caracterização do Problema:] Qual a pergunta a ser respondida? Quais
informações/recursos necessários na investigação?
	\item[Formulação da Hípotese:] Quais explicações possíveis para o que foi observado?
	\item[Previsão:] Dadas explicações [corretas] para as observações, quais os
	resultados previstos?
	\item[Experimentos:] \ \\\vspace{-2em}
		\begin{enumerate}
			\item Execute testes [reproduzíveis] da hipótese, coletando dados.
			\item Analise os dados.
			\item Interprete os dados e tire conclusões:
				\begin{itemize}
				\item que comprovam a hipótese;
				\item que invalidam a hipótese \emph{ou levam a uma nova hipótese}.
				\end{itemize}
		\end{enumerate}
	\item[Documentação:] Registre e divulgue os resultados.
	\item[Revisão de Resultados:] Validação dos resultados por outras pessoas
	[capacitadas].
\end{description}%

Geralmente se começa com a revisão sistemática, uma metodologia de pesquisa
específica para juntar e avaliar material relevante a determinado tópico~\cite{Biolchini_2005_Systematicreviewin}.

\subsection{Veja Também}
\begin{itemize}
	\item Google Acadêmico
		\\\url{http://scholar.google.com.br/}%
	\item ACM Digital Library
		\\\url{http://dl.acm.org/}%
	\item Portal \acrshort{CAPES}
		\\\url{http://www.periodicos.capes.gov.br/}%
	\item IEEE Xplore
		\\\url{http://ieeexplore.ieee.org/Xplore/home.jsp}%
	\item ScienceDirect
		\\\url{http://www.sciencedirect.com/}%
	\item Springer Link
		\\\url{http://link.springer.com/}%
\end{itemize}

Para buscar referências, \emph{The DBLP Computer Science Bibliography}\footnote{\url{http://dblp.uni-trier.de/}}
é um ótimo recurso. Veja o \refApendice{Apendice_Fichamento} para instruções
sobre como organizar as informações de artigos científicos.


%%%%%%%%%%%%%%%%%%%%%%%%%%%%%%%%%%%%%%%%%%%%%%%%%%%%%%%%%%%%%%%%%%%%%%%%%%%%%%%%
%%%%%%%%%%%%%%%%%%%%%%%%%%%%%%%%%%%%%%%%%%%%%%%%%%%%%%%%%%%%%%%%%%%%%%%%%%%%%%%%
%%%%%%%%%%%%%%%%%%%%%%%%%%%%%%%%%%%%%%%%%%%%%%%%%%%%%%%%%%%%%%%%%%%%%%%%%%%%%%%%
\section{\LaTeX}%

\TeX\ é ``\emph{a typesetting system intended for the creation of beautiful books
 - and especially for books that contain a lot of mathematics}''~\cite{Knuth_1986_texbook},
 um sistema de tipografia muito utilizado na produção de textos técnicos devido
 a qualidade final, principalmente das fórmulas e símbolos matemáticos gerados.

\LaTeX\ é um conjunto de macros para facilitar o uso de \TeX~\cite{lamport_latex:_1994},
cujos pacotes (a maioria centralizada na rede \acrshort{CTAN}~\cite{greenwade93}), oferecem
inúmeras possibilidades. Este sistema tipográfico visa explorar as potencialidades
da impressão digital, sem que o resultado seja alterado em função de diferenças
entre plataformas/sistemas.

Em uma publicação, um \emph{autor} entrega o texto a uma editor que define a
formatação do documento (tamanho da fonte, largura de colunas, espaçamento, etc.)
e passa as instruções (e o manuscrito) ao tipógrafo, que as executa. Neste processo,
\LaTeX\ assume os papéis de editor e tipógrafo, mas por ser ``apenas'' um programa
de computador, o autor deve prover algumas informações adicionais ~\cite{Oetiker_1995_notsoshort},
geralmente por meio de marcações (comandos).

Esta abordagem de linguagem de marcação (em que se indica como o texto deve ser
formatado) é diferente da abordagem OQVVEOQVO (``o que você vê é o que você
obtém\footnote{Do inglês WYSIWYG - ``What You See Is What You Get''.}'') de programas
para edição de texto tradicionais (como MS Word, LibreOffice Write, etc.).
Apesar destes programas serem extremamente úteis para gerar textos simples, que
são a grande maioria dos documentos, eles geralmente não têm a capacidade de lidar
corretamente com documentos complexos (como dissertações ou teses), conforme ilustrado
na \refFig{latexvsword}.%

\figuraBib{miktex}{\LaTeX\ vs MS Word}{pinteric_latex_2004}{latexvsword}{width=.45\textwidth}%

Existem diversas discussões quanto ao uso de editores de texto\footnote{Por exemplo:
\emph{Word Processors: Stupid and Inefficient} \url{http://ricardo.ecn.wfu.edu/~cottrell/wp.html}},
não há um consenso quanto a melhor forma de se gerar um documento de qualidade,
e a maioria das mídias científicas disponibiliza modelos para ambas.

Mas pode-se dizer que \LaTeX\ é mais indicado para:
\begin{itemize}
	\item notação matemática;
	\item referências cruzadas;
	\item separação clara entre conteúdo e formatação.
\end{itemize}

Enquanto os editores tradicionais são indicados para:
\begin{itemize}
	\item edição colaborativa (são mais populares);
	\item produção imediata (leve curva de aprendizado).
\end{itemize}

\subsection{Veja Também}
\begin{itemize}
	\item Introdução ao \LaTeX
		\\\url{http://latexbr.blogspot.com.br/2010/04/introducao-ao-latex.html}
	\item \LaTeX\ - A document preparation system
		\\\url{http://www.latex-project.org/}
	\item The \acrlong{CTAN}
		\\\url{http://ctan.org}
	\item \TeX Users Group
		\\\url{http://tug.org}
	\item \TeX\ - \LaTeX\ Stack Exchange
		\\\url{http://tex.stackexchange.com}
	\item \LaTeX\ Wikibook
		\\\url{http://en.wikibooks.org/wiki/LaTeX}
	\item write\LaTeX
		\\\url{http://www.writelatex.com}
\end{itemize}



%%%%%%%%%%%%%%%%%%%%%%%%%%%%%%%%%%%%%%%%%%%%%%%%%%%%%%%%%%%%%%%%%%%%%%%%%%%%%%%%
%%%%%%%%%%%%%%%%%%%%%%%%%%%%%%%%%%%%%%%%%%%%%%%%%%%%%%%%%%%%%%%%%%%%%%%%%%%%%%%%
%%%%%%%%%%%%%%%%%%%%%%%%%%%%%%%%%%%%%%%%%%%%%%%%%%%%%%%%%%%%%%%%%%%%%%%%%%%%%%%%
\section{Plágio}%
O JusBrasil\footnote{\url{http://www.jusbrasil.com.br}} define plágio como
``reprodução, total ou parcial, da propriedade intelectual de alguém, inculcando-se
o criador da idéia ou da forma. Constitui crime contra a propriedade imaterial
violar direito de autor de obra literária, científica ou artística.''


A \acrfull{CEP} da Presidência da República decidiu  ``pela
aplicação de sanção ética aos servidores públicos que incorrerem na prática de
plágio"\footnote{\url{http://www.comissaodeetica.unb.br/index.php?view=article&id=8:plagio-academico}},
e a \acrfull{CAPES} recomenda que se adote políticas de conscientização e informação
sobre a propriedade intelectual, baseando-se na Proposição 2010.19.07379-01,
referente ao plágio nas instituições de ensino\footnote{\url{https://www.capes.gov.br/images/stories/download/diversos/OrientacoesCapes_CombateAoPlagio.pdf}}.


\subsection{Veja Também}
\begin{itemize}
	\item IEEE Plagiarism FAQ
		\\\url{http://www.ieee.org/publications_standards/publications/rights/plagiarism_FAQ.html}
	\item Relatório da Comissão de Integridade de Pesquisa do CNPq
		\\\url{http://www.cnpq.br/web/guest/documentos-do-cic}
\end{itemize}

\fi

%\section{Normas CIC}
% \href{http://monografias.cic.unb.br/dspace/normasGerais.pdf}{Política de Publicação de Monografias e Dissertações no Repositório Digital do CIC}%
% \href{http://monografias.cic.unb.br/dspace/}{Repositório do Departamento de Ciência da Computação da UnB}

% \href{http://bdm.bce.unb.br/}{Biblioteca Digital de Monografias de Graduação e Especialização}