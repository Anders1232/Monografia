A tecnologia tem se tornado cada vez mais ubíqua na sociedade. Com o advento da internet a interação dos serem humanos com a tecnologia explodiu, gerando um tráfego imenso de dados e com isso a necessidade de processamento em larga escala. Com isso surgiu o conceito de Nuvem Computacional, um paradigma que permite processamento em larga escala sem ser necessário que o usuário tenha em mãos hardware com tamanha capacidade computacional. Afinal, como isso é possível?




\section{Objetivos}
Este trabalho tem como objetivo:
\begin{itemize}
	\item Implementar um escalonador capaz de escalonar tarefas para arquiteturas heterogêneas no BioNimbuZ.
	\item Testar o ganho de desempenho obtido ao se utilizar o escalonador desenvolvido na nuvem.
\end{itemize}

\section{Estrutura do Trabalho}
Esse projeto contém mais três capítulos e um apêndice. O segundo capítulo aborda a atividade de escalonamento e propõe o algoritmo que será implementado. O capítulo três descreve a plataforma de nuvens Federadas BioNimbuZ. E o quarto e último capítulo fala como foi o processo de implementação do escalonador. Também haverá um apêndice falando sobre software livre.
