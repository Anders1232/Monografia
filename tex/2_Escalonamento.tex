A atividade de escalonamento pode ser otimizada para vários objetivos, entre os quais podemos citar:
\begin{itemize}
	\item Maximinizar quantidade de trabalho realizada por unidade de tempo;
	\item Minimizar tempo no qual trabalhos ficam esperando para serem executados;
	\item Minimizar tempo entre um conjunto de trabalhos estarem prontos para serem executados até o fim da execução do conjunto(latência ou tempo de resposta);
	\item Distribuir de forma justa o tempo que cada um dos trabalhos terão de uso de um recurso escasso.
\end{itemize}

Esses objetivos são, às vezes, contraditórios. Na prática prioriza-se um conjunto de métricas como base para otimização. Por exemplo o 