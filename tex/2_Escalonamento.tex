O escalonamento é um problema clássico da computação. Que surgiu junto com os sistemas operacionais multitarefas. Sua complexidade é considerada NP-difícil, porém isso não impediu a evolução dos sistemas operacionais, cujos escalonadores são desenvolvidos com objetivos e metodologias diferentes para alcançarem esses objetivos. De qualquer forma um problema que todos os sistemas operacionais multitarefa buscam resolver é o \textit{starvation}, que é quando um processo/\textit{thread} permanece em execução por muito tempo, impedindo que outros sejam executados.

\section{Tipos de escalonadores}

Escalonadores podem ser categorizados de várias formas, com base nas funcionalidades que possui, eis alguns exemplos:

\begin{itemize}
	\item \textbf{O que é escalonado}: Podem ser:
	\begin{itemize}
		\item processos, \textit{threads} para serem executados;
		\item jobs para serem excecutados na nuvem;
		\item páginas da memória \acrfull{RAM} para \textit{swapping};
		\item pacotes para serem transmitidos pela rede;
		\item requisições de leitura/escrita em memória secundária.
	\end{itemize}
	\item \textbf{Suporte para preempção}: Se permite interrupção de uma tarefa em execução para atender outras;
	\item \textbf{Suporte para prioridade}: Se permite que priorização entre as tarefas que serão escalonadas;
	\item \textbf{Suporte para para tempo-real}: Se permite que requerem execução constante tenham esse requisito atendido.
\end{itemize}


O problema do escalonamento é o método pelo qual trabalho, definido por algum conjunto de características(como duração e requisitos), é atribuído à recursos que capazes de completá-lo. Por mais que exista a dificuldade teórica\cite{ULLMAN1975384}, isso não impediu a evolução dos Sistemas Operacionais, os quais foram capazes de fazer escalonamento de vários processos mesmo na época de \acrfull{CPU}s tinham apenas um núcleo. Popularizando dessa forma os computadores pessoais na década de 80.
Atualmente, as \acrshort{CPU}s possuem vários núcleos, e são capazes de ter mais de um contexto carregado por núcleo, vide Ryzen™ Threadripper™\cite{Ryzen}. O que faz com que seja necessário que o processo de escalonamento leve em consideração como o mesmo será distribuído(ou não) entre os núcleos.

A atividade de escalonamento pode ser otimizada para vários objetivos, entre os quais podemos citar:
\begin{itemize}
	\item Maximinizar quantidade de trabalho realizada por unidade de tempo;
	\item Minimizar tempo no qual trabalhos ficam esperando para serem executados;
	\item Minimizar tempo entre um conjunto de trabalhos estarem prontos para serem executados até o fim da execução do conjunto(latência ou tempo de resposta);
	\item Distribuir de forma justa o tempo que cada um dos trabalhos terão de uso de um recurso escasso.
\end{itemize}

Esses objetivos são, às vezes, contraditórios. Na prática prioriza-se um conjunto de métricas como base para otimização. Por exemplo o GNU/Linux utiliza o \acrfull{CFS}, que se baseia no algoritmo \textit{Fair queuing}. Como o nome já diz, o foco desse escalonador está em ser justo. Internamente utiliza-se uma árvore rubro-negra indexados pelo tempo gasto no processador. Para ser justo, o tempo máximo de cada processo fica em execução interrupta é o quociente do tempo que o processo ficou aguardando para ser executado pelo número total de processos.

\section{Escalonamento em Nuvens Computacionais}

Este trabalho focará no escalonamento em federações de nuvens computacionais. O qual recentemente presenciou a ascensão do uso de unidades de processamento gráfico(\acrfull{GPU}) para processamento de propósito geral (\acrfull{GPGPU}\cite{Dimitrov:2009:USA:1513895.1513907}\cite{Yang:2010:GCM:1809028.1806606}. Nesse contexto, o problema do escalonamento pode ser formalmente definido da seguinte forma:\\
Dado:
\begin{itemize}
	\item Conjunto $T$ de tarefas;
	\item Conjunto $M$ de máquinas virtuais;
	\item Conjunto $C, |C| = |M|$ de CPUs das máquinas virtuais;
	\item Conjunto $G,  |G| \le |M|$ de GPUs das máquinas virtuais;
	\item Função $F: T \times (C \cup G) \to \mathbb{R}$, o qual estima o tempo de execução da tarefa $T_{i}$ no recurso designado.
\end{itemize}
Encontrar uma função injetora $A: T \to C \cup G$, que minimize $\sum_{t \in T} F(t, A(t) )$.\\

Por mais que o problema do escalonamento seja um clássico na área da Ciência da Computação, são poucos que lidam com escalonamento em nuvens computacionais, e ainda menos que lidam com nuvens compostas por arquiteturas heterogêneas. 

/*colocar aqui revisão do estado da arte de computação em nuvem*/
Gouasmi\cite{MapReduce_sched_8034997} apresenta um algoritmo de \textit{MapReduce}\cite{Dean:2008:MSD:1327452.1327492} para escalonamento em nuvens federadas que foca em priorizar a execução de \textit{jobs} em nuvens/máquinas virtuais que já contém os dados necessários para a execução, com o objetivo de evitar transferências desnecessárias na rede. O algoritmo proposto é completamente distribuído e melhor que o \textit{MapReduce} anteriormente utilizado, porém nada é dito sobre escalonamento para máquinas com arquiteturas heterogêneas.

\section{Algoritmo proposto}

O escalonador proposto para implementação segue a ideia básica de escalonamento de listas. Haverão três listas: lista de tarefas a serem feitas, lista de \acrshort{CPU}s disponíveis e lista de \acrshort{GPU}s disponíveis.
A lista de tarefas é ordenada por tempo previsto de execução, que é dado por um estimativa a partir do programa a ser rodado e do arquivo de entrada. Essa ordenação será em ordem decrescente de tempo previsto. A lista de \acrshort{CPU}s disponíveis é ordenada com base na frequência e no número de núcleos. A lista de \acrshort{GPU}s tem sua ordem determinada pela quantidade de operações em pontos flutuante consegue realizar por segundo.

O escalonamento ocorre da seguinte forma, como ilustrado na figura \ref{Escalonamento} :\\
\newline
\iffalse
\begin{enumerate}
	\item Existem tarefas que podem ser executadas nos recursos disponíveis?
	\item Se sim, obtenha o processo que está no topo da lista.
	\begin{enumerate}
		\item É capaz de rodar em GPU?
		\item Se sim:
		\begin{enumerate}
			\item O tempo prevista para rodá-lo na melhor \acrshort{GPU} disponível é melhor que o tempo previsto para rodá-lo na melhor \acrshort{CPU} disponível?
			\item Se sim, escalone-o para nessa \acrshort{GPU}.
			\item Se não, escalone-o para na melhor \acrshort{CPU} disponível.
		\end{enumerate}
		\item Se não, escalone-o na melhor \acrshort{CPU} disponível.
		\item Remova a tarefa e o recurso alocado de suas respectivas listas.
	\end{enumerate}
	\item Se não, encerre o escalonamento.
	\item Volte para a regra 1.
\end{enumerate}
\fi
\begin{algorithm}
\caption{Escalonamento heterogêneo baseado em listas}
\begin{algorithmic}
	\Procedure{Escalonar}{listas lTarefas, lCPUs e lGPUs}
	\While{Existem tarefas que podem ser executadas nos recursos disponíveis?}
		\State{$aux \gets lTarefas[0] $}
		\If{aux é capaz de rodar em GPU}
			\If{$T_{previsto}(lGPUs[0]) < T_{previsto}(lCPUs[0])$}
			\State{Escalone\ aux\ para\ lGPUs[0]}
			\Else
				\State{Escalone\ aux\ para\ lCPUs[0]}
			\EndIf
		\Else
			\State{Escalone\ aux\ para\ lCPUs[0]}
		\EndIf
		\State{Remova a tarefa e o recurso alocado de suas respectivas listas.}
	\EndWhile
	\State{Encerra escalonamento}
\EndProcedure
\end{algorithmic}
\end{algorithm}


\begin{figure}[htbp]
%	\centerline{\includegraphics[scale=0.04]{img/EscalonadorProposto.png}}
	\centerline{\includegraphics[scale=0.032]{img/EscalonadorProposto2.png}}
	\caption{Diagrama de Funcionamento do Algoritmo de Escalonamento.}
	\label{Escalonamento}
\end{figure}


Observa-se que, para o algoritmo supracitado seja válido, pressupõe-se, que toda tarefa do algoritmo é capaz de rodar em CPU. O que é correto, pois atualmente todas as tarefas do BioNimbuZ rodam em \acrshort{CPU}. O pressuposto simplifica o primeiro passo do algoritmo, pois se a lista de \acrshort{CPU}s não estiver vazia, essa condição é automaticamente satisfeita. Futuramente pode ser necessário adaptá-lo para ser capaz de lidar com tarefas que só são capazes de serem executadas em \acrshort{GPU}.

Como trabalho futuro, se o BioNimbuZ obtiver suporte executação de uma mesma tarefa de forma distribuída, uma pequena modificação que pode ser feita no algoritmo supracitado é: ao invés de remover as tarefas escalonadas da lista de tarefas, colocá-las no fim dessa mesma lista. Fará com que o escalonamento seja interrompido apenas quando todos os recursos disponíveis forem alocados, pois sempre haverá tarefas para serem alocadas.

