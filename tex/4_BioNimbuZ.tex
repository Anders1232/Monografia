Este capítulo possui como objetivo apresentar a plataforma de nuvens federadas BioNimbuZ, que foi selecionada para desenvolver o escalonador para plataformas heterogêneas proposto neste trabalho. A primeira seção apresenta de forma geral as principais características do BioNimbuZ, e descreve sua evolução ao longo do tempo. A segunda seção descreve a arquitetura do BioNimbuZ e seus principais componentes. A última seção foca sobre os escalonadores que existem atualmente no BioNimbuZ.

\section{Visão Geral}

O BioNimbuZ é uma plataforma livre de nuvens federadas para execução de \textit{workflows}, desenvolvido no \acrfull{LABID} por alunos de graduação e pós-graduação. Ele foi originalmente proposto por Saldanha\cite{Saldanha_BioNimbus} e refinado por alunos de iniciação científica, graduação, mestrado e doutorado\cite{BioNimbuZ_Breno_Deric} \cite{BioNimbuZ_6846526} \cite{6732620_BioNimbuZ_ACOsched} \cite{BioNimbuZ_Willian_C99} \cite{closer12_BioNimbuZ_AHP} \cite{BioNimbuZ_Vegara}.

É possível integrar nuvens de diversos tipos de governança (vide Capítulo 2), o que permite que cada provedor mantenha suas políticas e características internas, e isso reduz a dependência da plataforma e, consequentemente, dos usuários de dependência de servidores específicos de nuvem. Essa funcionalidade é alcançada graças à facilidade e flexibilidade na inclusão de novos provedores na plataforma, o qual é regida por utilização de \textit{plugins} de integração, que traduzem requisições vindas da plataforma para comandos equivalentes específicos de cada servidor. Isso é fundamental para evitar \textit{vendor lock-in}\cite{VendorLockInDef}, que é a dependência de um serviço a um provedor em específico, por mais que existam outros.

Em sua concepção inicial, a plataforma BioNimbuZ foi implementada toda por meio de comunicação \acrfull{P2P}, porém, durante sua evolução optou-se por integrar o Zookeeper\cite{Zookeeper} à plataforma para gerenciar os serviços distribuídos. %Ainda assim, características que Saldanha propôs\cite{Saldanha_BioNimbus}, por exemplo tolerância a falhas, suporte a vários provedores de nuvens, elevado poder de armazenamento e processamento, foram não só mantidos, como também expandidos.

Moura \textit{el al.}\cite{BioNimbuZ_6846526} desenvolveram a Chamada de Procedimento Remoto \acrshort{RPC}\cite{RPC_1701928} para a plataforma. Fazendo uso do Apache Avro\cite{Avro}, que possibilita comunicação de forma transparente entre computadores, para que seja possível a chamada de procedimentos em outros computadores via rede. Além disso, o núcleo do BioNimbuZ foi refatorado para utilização do Apache ZooKeeper\cite{Zookeeper}, e também foi implementada uma política de armazenamento que considera latência e o local em que o serviço será executado, utilizando \acrfull{SFTP} para transferência de arquivos.

Barreiros \textit{el al.}\cite{BioNimbuZ_BioCirrus} refinaram a política de armazenamento, que analisa a viabilidade de compactação de arquivos pré-transferência com base na largura de banda, tempo de transferência de arquivos e tempo de compressão e descompressão.

Ramos \textit{el al.}\cite{BioNimbuZ_Ramos} desenvolveram um controlador de \textit{jobs} para comunicação entre o núcleo do BioNimbuZ e a Camada de Interface com o Usuário, além de desenvolver uma interface gráfica que tornou a plataforma mais acessível.

Novamente, Barreiros\cite{BioNimbuZ_Willian_C99} desenvolveu o escalonador que se baseia no \textit{beam search} interativo multiobjetivo, chamado C99, com o objetivo de aumentar a eficiência do escalonamento, que passou a levar em consideração o custo por hora dos recursos a serem alocados.

Santos\cite{BioNimbuZ_Santos} refinou o serviço de armazenamento do BioNimbuZ, adicionando suporte ao uso de serviços de armazenamentos providos pelos diferentes provedores de nuvem, melhorando a praticidade e a usabilidade da plataforma.

Ao longo dos trabalhos supracitados, que evoluíram o sistema proposto inicialmente por Saldanha\cite{Saldanha_BioNimbus}, o BioNimbuZ adquiriu uma arquitetura em camadas, que será descrita na próxima seção.

\section{Arquitetura}
O BioNimbuZ foi implementado por meio de uma arquitetura em camadas, dispostas de forma hierárquica e distribuída. Ele possui quatro camadas principais: Aplicação, Integração, Núcleo e Infraestrutura, como apresentado na Figura \ref{Arquitetura}.

Internamente, o BioNimbuZ utiliza o Apache Zookeeper\cite{Zookeeper} para prover serviços de coordenação de ambientes distribuídos. O Apache Zookeeper foi desenvolvido pela fundação Apache\cite{Apache}, tem como objetivo ser de fácil manuseio. Ele possui um modelo de dados semelhante a uma estrutura de diretórios.

Uma outra tecnologia que também é utilizada no BioNimbuZ é o Apache Avro\cite{Avro}, para serialização de dados para transmissão pela rede.


\begin{figure}[htbp]
	%	\centerline{\includegraphics[scale=0.04]{img/EscalonadorProposto.png}}
	\centerline{\includegraphics[scale=0.45]{img/ArquiteturaBioNimbuZ.png}}
	\caption{Arquitetura do BioNimbuZ\cite{Alves_BioNimbuz}.}
	\label{Arquitetura}
\end{figure}



\subsection{Camada de Aplicação} Esta camada é responsável por prover a interface de comunicação com o usuário, seja via uma interface gráfica (GUI), seja via \textit{web}. Após fazer \textit{login}, o usuário pode submeter \textit{workflows} para serem executados e fazer \textit{upload} dos arquivos necessários. Além de poder acompanhar o andamento de seus \textit{workflows} e obter, caso queira, os resultados que já tiverem sido produzidos.

\subsection{Camada de Integração} A Camada de Integração tem como objetivo integrar as Camadas de Aplicação e de Núcleo, fazendo uso do \textit{framework} \acrshort{REST} para prover de forma prática essa funcionalidade, utilizando operações definidas no protocolo \textit{HTTP}, como \textit{GET}, \textit{DELETE} e \textit{PUT}.
Existem três tipos de mensagens trocadas entre o Núcleo e a Camada de Aplicação:
\begin{itemize}
	\item \textit{Request}: Requisições da camada de Aplicação que contém todos os dados necessários para o seu processamento;
	\item \textit{Response}: Respostas que definem as mensagens enviadas da camada de Núcleo do BioNimbuZ;
	\item \textit{Action}: Comandos a serem executados pelo núcleo, que são uma requisição enviada ao núcleo para se obter dados na resposta.
\end{itemize}
	
	\subsection{Camada de Núcleo} A Camada de Núcleo é a camada mais importante do BioNimbuZ. Ela realiza toda a gerência da federação, provendo vários serviços. Entre eles:

	\begin{itemize}
		\item Serviço de Predição: Objetiva orientar o usuário do BioNimbuZ a escolher as melhores combinações de máquinas virtuais/provedores a partir da especificação do \textit{workflow} a ser executado e do custo pretendido;
		\item Serviço de Tarifação: Responsável por calcular o valor que os usuários devem pagar pelos serviços provindos do BioNimbuZ. Para tal, comunica-se com o serviço de monitoramento para obter informações tais como tempo de execução e quantidade de máquinas virtuais alocadas. É função deste serviço garantir o cumprimento das métricas de tarifação das nuvens integradas à federação, e repassar o valor ao usuário;
		\item Serviço de Segurança: Realiza, principalmente, a autenticação de usuário, além de verificar as autorizações do mesmo. Contudo, muitos outros aspectos de segurança computacional podem ser implementados por este serviço, tais como criptografia na troca de mensagens;
		\item Serviço de Tolerância a Falhas: Como o nome diz, este serviço é responsável em certificar que todos os serviços do BioNimbuZ estejam disponíveis o máximo de tempo possível. Além de ter a responsabilidade de tratar quaisquer falhas que venham a ocorrer, tira vantagem da arquitetura distribuída do BioNimbuZ para prover redundância de dados;
		\item Serviço de Armazenamento: Possui a responsabilidade de gerenciar arquivos utilizados como entrada e/ou saída de cada estágio de um \textit{workflow}. Deve desempenhar seu papel de forma eficiente, do ponto de vista de custos de armazenamento e transmissão desses dados entre o local que está armazenado e o local que serão processados;
		\item Serviço de Elasticidade: Tem como objetivo manipular, dinamicamente, as máquinas virtuais que estão associadas à um tarefa, com o objetivo de otimizar o uso dos recursos, como também, aumentar a capacidade de processamento de equipamentos, além de poder alocar mais de uma máquina para uma tarefa.
		\item Serviço de Descoberta: É o responsável por identificar os prevedores de nuvem disponíveis, e armazenar informações sobre eles tais como latência de rede, capacidade de armazenamento e processamento, além de informações sobre arquivos de entrada e saída dos \textit{workflows} e argumentos para execução das \acrshort{VM}s.
		\item Serviço de Escalonamento: Responsável por fazer o escalonamento dos \textit{jobs} submetidos ao BioNimbuZ. Não é responsabilidade do Serviço de Escalonamento lidar com dependências, pois essa responsabilidade é do Controlador de \textit{Jobs}. Assim o Serviço de Escalonamento só fará a distribuição dos \textit{jobs} que estiverem prontos para serem executados;
		\item Serviço de Monitoramento: Responsável por monitorar o estado das \acrshort{VM}s instanciadas pela plataforma, e disponibilizar essa informação ao usuário;
		\item Gerenciador de Persistência: Tem como objetivo organizar o armazenamento de arquivos gerados durante a execução dos \textit{workflows}, além de \textit{uploads} de arquivos que serão utilizados como entrada em futuros \textit{workflows};
		\item Controlador de \textit{Jobs}: Responsável por gerir \textit{jobs}, em especial, os que possuem dependência ficam à espera até poderem ser escalonados.

	
	\end{itemize}
	
	\subsection{Camada de Infraestrutura} A Camada de Infraestrutura disponibiliza uma interface de comunicação do BioNimbuZ com os provedores de nuvem. Esta camada utiliza \textit{plugins} para mapear requisições provenientes da Camada de Núcleo para comandos específicos de cada provedor.

O BioNimbuZ é capaz de ser integrado tanto à nuvens públicas quanto privadas, utilizando \textit{plugins} para permitir conexão com vários provedores de nuvem, cada qual com sua própria interface. Os \textit{plugins} não existem apenas na camada de integração, pois vários serviços da camada de núcleos também são disponibilizados como tal, provendo grande flexibilidade à plataforma.

Diante do objetivo deste trabalho, o Serviço de Escalonamento do BioNimbuZ será detalhado na próxima seção.

\section{Serviço de Escalonamento}

Em suas primeiras fases de desenvolvimento, o escalonador do BioNimbuZ apenas realizava uma associação \acrfull{FIFO} entre \textit{jobs} disponíveis e as \acrshort{VM}s instanciadas. Em seguida, esta política de escalonamento evoluiu para um escalonamento \textit{Round-Robin}. Desde então, foram propostos vários escalonadores diferentes, que serão explicados a seguir.

\subsection{\textit{Analytic Hierarchy Proccess}}
Em sua primeira versão oficial, o BioNimbuZ possuía um escalonador baseado no \acrfull{AHP}\cite{6732620_BioNimbuZ_ACOsched}, uma técnica de análise de decisões complexas, de amplo uso no meio corporativo. O problema é quebrado em uma estrutura hierárquica que cada elemento da hierarquia interage apenas com níveis imediatamente superiores e inferiores. Após análise comparativa de como cada elemento interfere em todos os outros elementos dos níveis hierárquicos vizinhos, é obtido o resultado da análise, que revela as opções escolhidas com base nos critérios analisados.

\subsection{\textit{Ant Colony Optmization}}
Esta solução é baseada na meta-heurística \acrfull{ACO}\cite{ACO_DORIGO2005243} para o problema de otimização combinatória difícil. Esta meta-heurística foi inspirada no comportamento das formigas enquanto andam. Oliveira \textit{el al.}\cite{6732620_BioNimbuZ_ACOsched} desenvolveram e implementaram um escalonador para o BioNimbuZ com esta proposta de solução, que obteve resultados positivos comparados ao escalonador anterior, baseado no \acrshort{AHP}.

\subsection{\textit{Beam Search} Multiobjetivo}
O escalonador utilizado atualmente, desenvolvido por Barreiros\cite{BioNimbuZ_Willian_C99}, é um escalonador cuja proposta principal é ser interativo e multiobjetivo, ou seja, que pode ser interrompido durante sua execução e ainda assim retornar um resultado válido. Este algoritmo considera, durante o processamento, a capacidade das máquinas virtuais disponíveis e o preço de cada uma delas. O escalonamento ocorre em três etapas, as quais são:
\begin{enumerate}
	\item Busca gulosa sobre conjuntos de Pareto, ou seja, conjuntos de resultados considerados equivalentes dado um conjunto de critérios, para poda (remoção) de conjuntos soluções não ideias;
	\item \textit{Limited Discrepancy Search}\cite{Harvey:1995:LDS:1625855.1625935}, cujo objetivo é impedir que a poda (remoção) do conjuntos soluções que pode levar a um conjunto solução melhor futuramente; e
	\item \textit{Beam Search} Iterativo, que busca iterativamente os melhores conjuntos resultado.
\end{enumerate}

Uma funcionalidade que não foi abordada nos escalonadores citados acima é que eles não possuem suporte para executar \textit{jobs} em \acrshort{GPU}s, apenas em \acrshort{CPU}s. Diante do exposto, o objetivo deste trabalho é considerar no escalonamento as máquinas com \acrshort{CPU}s e \acrshort{GPU}s. O próximo capítulo discorrerá sobre escalonadores, e apresentará o algoritmo de escalonamento proposto neste trabalho.