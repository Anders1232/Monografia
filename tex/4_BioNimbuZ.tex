O BioNimbuZ é uma plataforma livre de nuvens confederadas para execução de workflows de bioinformática desenvolvida no laboratório de Bioinformática e Dados(LABID) por alunos de graduação e pós-graduação. Originalmente proposta por Saldanha\cite{Saldanha_BioNimbus} e refinada por alunos de iniciação científica, graduação, mestrado e doutorado.\cite{closer12_BioNimbuZ_AHP}\cite{BioNimbuZ_6846526} \cite{6732620_BioNimbuZ_ACOsched}\cite{BioNimbuZ_Breno_Deric}\cite{BioNimbuZ_Vegara}\cite{BioNimbuZ_Willian_C99}

Implementado utilizando uma arquitetura de camadas, o BioNimbuZ possui 4 camadas, descritas a seguir:
\begin{itemize}
	\item Camada de Aplicação: Responsável por prover a interface de comunicação com o usuário, seja via uma interface gráfica(GUI), seja via \textit{web}. Após fazer login, o usuário pode enviar \textit{workflows} para serem executados e fazer \textit{upload} do arquivos necessários. Além de poder acompanhar o andamento de seus \textit{workflows} e pode obter, caso queira, o resultados parciais que já tiverem sido produzidos.
	
	\item Camada de Integração: Tem como objetivo de integrar as Camadas de Aplicação e de Núcleo, fazendo uso do \textit{framework} \acrshort{REST} para prover de forma prática essa funcionalidade, utilizando operações definidas no protocolo \textit{HTTP}, como \textit{GET}, \textit{DELETE} e \textit{PUT}.
	Existem três tipos de mensagens trocadas entre o Núcleo e a camada de Aplicação:
	\begin{itemize}
		\item \textit{Request}: Requisições da camada de Aplicação que contém todos os dados necessários para o seu processamento;
		\item \textit{Response}: Respostas que definem as mensagens enviadas da camada de Núcleo do BioNimbuZ; e
		\item \textit{Action}: Comandos a serem executados pelo núcleo, que são uma requisição enviada ao núcleo para se obter dados na resposta.
	\end{itemize}
	
	\item Camada de Núcleo: Realiza toda a gerência da federação, provendo vários serviços. Entre eles:
	\begin{itemize}
		\item Serviço de predição: Objetiva orientar o usuário do BioNimbuZ a escolher as melhores combinações de máquinas virtuais/provedores a partir da especificação do \textit{workflow} a ser executado e custo pretendido;
		\item Serviço de tarifação: Responsável por calcular o valor que os usuários devem pagar pelos serviços providndos do BioNimbuZ. Para tal, comunica-se com o serviço de monitoramento para obter informações como tempo de execução e quantidade de máquinas virtuais alocadas. É função desse serviço garantir o cumprimento das métricas de tarifação das nuvens integradas à federação, e repassar o valor ao usuário;
		\item Serviço de segurança: Realiza principalmente a autenticação de usuário, além de verificar as autorizações do mesmo. Contudo muitos outros aspectos de segurança computacional podem ser implementados por esse serviço, como criptografia na troca de mensagens;
		\item Serviço de Tolerância a Falhas: Como o nome diz, esse serviço é responsável em certificar que todos os serviços do BioNimbuZ estejam disponíveis o máximo de tempo possível. Além de ter a responsabilidade de tratar quaisquer falahas que venham a ocorrer. Tira vantagem da arquitetura distribuída do BioNimbuZ para prover redundância de dados;
		\item Serviço de Armazenamento: Posui a responsabilidade de gerenciar arquivos utilizados como entrada e/ou saída de cada estágio de um \textit{workflow}. Deve desempenhar seu papel de forma eficiente, do ponto de vista de custos de armazenamento e transmissão desses dados entre o local que está armazenado e o em que serão processados;
		\item Serviço de Escalonamento: Responsável por fazer o escalonamento de curto prazo de \textit{jobs}. Com curto prazo quer-se dizer que é tão somente o escalonamento de \textit{jobs} que estão prontas para serem executadas, à máquinas virtuais que os processarão. Não é responsabilidade do serviço de escalonamento lidar com dependências, pois essa responsabilidade é do controlador de \textit{jobs}, que será explicado em breve.
		No momento do início deste projeto, o BioNimbuZ possuia 5 políticas de escalonadonamento, são eles:
		\begin{itemize}
			\item AcoSched: Baseado em \textit{Load Balancing Ant Colony Scheduling}, desenvolvido por Oliveira. Baseado em heurística\cite{6732620_BioNimbuZ_ACOsched}
			\item AHP: Baseado em \textit{Analytic Hierarchy Process}\cite{6732620_BioNimbuZ_ACOsched}
			\item \textit{BasicSched}: Política \textit{First In First Out}, implementada no início da plataforma;
			\item C99: Baseia-se no \textit{Beam Search} interativo multiobjetivo. \cite{BioNimbuZ_Willian_C99} Esse é o escalonador em uso atualmente.
			\item \textit{RoundRobin}: O clássico escalonamento \textit{Round Robin};
		\end{itemize}
	
	\end{itemize}
	
	\item Camada de Infraestrutura: Disponibiliza uma interface de comunicação do BioNimbuZ com os provedores de nuvem. Utilizando \textit{plugins} para mapear requisições provenientes da Camada de Núcleo para comandos específicos de cada provedor.
\end{itemize}

O BioNimbuZ é capaz de ser integrado tanto à nuvens públicas quanto privadas, utilizando \textit{plugins} para permitir conexão com vários provedores de nuvem, cada qual com sua própria interface. Os \textit{plugins} não existem apenas na camada de integração: vários serviços da camada de núcleos também são disponibilizados como tal, provendo grande flexibilidade à plataforma.

Internamente, o BioNimbuZ utiliza o Apache Zookeeper\cite{Zookeeper} para prover serviços de coordenação de ambientes distribuídos. Desenvolvido pela fundação Apache\cite{Apache}, tem como objetivo ser e fácil manuseio. Possui um modelo de dados semelhante a uma estrutura de diretórios
Uma outra tecnologia que também é utilizada no BioNimbuZ é o Apache Avro\cite{Avro}, para serialização de dados para transmissão pela rede.
