Este capítulo apresenta, inicialmente a definição de nuvem computacional, suas vantagens e desvantagens. Em seguida, busca-se classificar os tipos de nuvem, tanto em termos de quais serviços são providos, quanto em termos de como a infraestrutura computacional é implementada. Por fim, é apŕesentado o conceito de federação de nuvens e como essas federações podem combinar provedores de nuvens distintos em prol da performance e da relação custo-benefício.

\section{Definição de Computação em Nuvem}

Por alguns anos a computação em nuvem não possuía uma definição bem definida principalmente por ser uma tecnologia nova. Muitas vezes sendo confundida com computação em \textit{Grid}. Atualmente sua definição já está bem definida, com duas principais definições. O \acrfull{NIST} define computação em nuvem como um modelo de computação distribuída com as seguintes características \cite{NIST_CLOUD_DEFINITION}: 
\begin{itemize}
	\item \textit{On demand self-service};
	\item Fácil acesso via rede;
	\item Recursos virtualizados;
	\item Rápida eslasticidade; e,
	\item Serviço mensurado.
\end{itemize}

Em contrapartida, Vaquero\cite{Vaquero:2008:BCT:1496091.1496100_Cloud_definition} define nuvens como um grande conjunto de recursos virtualizados,\iffalse tais como \textit{hardware},plataforma de desenvolvimentos e/ou serviços \fi de fácil acesso e uso; que podem ser dinamicamente reconfigurados para se ajustarem para uma variável, permitindo também um uso ótimo dos recursos. Esse conjunto de recursos são geralmente explorados por um modelo de uso \textit{pay-per-use} no qual garantias são providas pelo provedor da infraestrutura por meio de \acrfull{SLA}.

Analisando as definições supracitadas, percebe-se que ambas concordam que recursos computacionais, como armazenamento, capacidade de processamento, memória e rede são disponibilizados de forma virtualizada, acessível via rede, na qual o uso desses recursos é monitorado, geralmente com intenção de cobrança. Além de ser possível redimensionar os recursos disponíveis para uso \textit{on-the-fly}, funcionalidade denominada eslasticidade. Que ocorre de várias formas \cite{Coutinho2015}: seja redimensionando uma \acrshort{VM} em execução, ou seja criando novas máquinas virtuais para ajudar a prover o serviço, geralmente associadas a um distribuidor de carga.

Analisando o conjunto de características supracitadas é possível visualizar as vantagens dessa tecnologia: reduz o investimento inicial de empresas em infraestrutura de TI; a escalabilidade remove o custo de reinvestimento na infraestrutura, além de não desperdiçar investimentos feitos em servidores em momentos de pouca carga, se paga por aquilo que se consome. Isso tornou as nuvens computacionais muito populares e um bom negócio para as grandes empresas da área de \acrfull{TI}, que são capazes de prover infraestrutura computacional, como a \textit{Oracle}, \textit{Microsoft}, \textit{IBM}, \textit{Google} entre outras.

\section{Tipos de Nuvens}

Os mesmos autores que definem nuvem computacional também buscam categorizar tipos de nuvens, sobre a perspectiva de tipos de serviços oferecidos, existem três classificações para nuvem computacional, que são \cite{NIST_CLOUD_DEFINITION} \cite{Vaquero:2008:BCT:1496091.1496100_Cloud_definition}:

\begin{itemize}
	\item \textbf{\textit{Infrastructure as a Service:}} Também chamado de \textit{Metal as a Service:}, o provedor disponibiliza recursos computacionais virtualizados diretamente, em termos de máquinas virtuais. Capazes de serem reconfigurados dinamicamente pelo serviço de elasticidade, por exemplo a \textit{Oracle Cloud} \cite{OracleCloud};
	\item \textbf{\textit{Plataform as a Service:}} Uma plataforma sobre a qual os usuários podem implantar softwares ou serviços é disponibilizada. Os programas implanatados podem fazer uso de algumas funcionalidade fornecidas em forma de bibliotecas, linguagens de programação, entre outros. Um exemplo básico é o serviço de hospedagem de sites, que são disponibilizados, por exemplo, no \acrfull{AWS} e no \acrfull{GCP};
	\item \textbf{\textit{Software as a Service:}} O usuário tem acesso a softwares que estão sendo executados na nuvem computacional. O que permite migrar requisito computacional do equipamento onde o serviço executa para a nuvem. Um exemplo bastante conhecido é o \textit{Google Docs} \cite{GoogleDocs}.
\end{itemize}

A categorização de nuvens computacionais por meio dos serviços providos não é a única forma de classificar nuvens. Elas também são categorizadas de acordo com a forma em que são implantadas. Podendo ser públicas, privadas, híbridas e comunitárias, descritas abaixo\cite{NIST_CLOUD_DEFINITION}:

\begin{itemize}
	\item \textbf{Nuvem pública}: Criada para uso pelo público em geral. Podendo ser disponibilizada e/ou mantida por uma organização com ou sem fins lucrativos, 
	\item \textbf{Nuvem privada}: Quando a infraestrutura da nuvem é provida por uma única organização, para uso interno. Ela pode ser gerenciada e/ou operada por um terceiro. Nuvens privadas são uma solução para o problema de privacidade dos dados, pois ela pode estar disponível apenas internamente na organização.
	\item \textbf{Nuvem comunitária}: Nuvem com infraestrutura desenvolvida com o objetivo de ser utilizada por um grupo de pessoas e/ou organizações para fins comuns. Ela pode ser gerenciada por uma ou mais partes dos interessadas na nuvem.
	\item \textbf{Nuvem híbrida}: Combinação de duas ou mais das opções anteriores. Geralmente unificadas por meio de padronizações de protocolos de comunicação ou uso de tecnologia proprietária.
\end{itemize}

\section{Federações de Nuvens}

Como cada provedor de nuvem possui seu próprio método de precificação, uma interface própria para comunicação com os usuários de seus serviços e funcionalidades próprias, surgiu federações de nuvens computacionais com o objetivo de minimizar dependência entre os provedores do serviço e também reduzir custos, aproveitando o melhor de cada provedor de nuvem.

Existem duas formas nas quais plataformas de federação podem combinar os serviços das nuvens: horizontalmente e verticalmente\cite{5557976}\cite{7835207}. No modelo vertical instâncias de nuvem de um provedor A são criadas a partir de instâncias de nuvem de um provedor B, que pode solicitar serviços de um provedor de nuvem C, e assim por diante. No modelo horizontal as nuvens trabalham sem existir uma hierarquia entre si, assemelhando-se bastante com sistemas de rede \textit{peer to peer}. É possível combinar mais de um estilo em uma mesma federação.

%\subsection{Revisão bibliográfica}

Buyya \cite{Buyya:2010:IUF:2143583.2143586} descreve uma arquitetura para federações de nuvens que faz uso de uma \textit{exchange} para aproximar clientes de provedores de nuvem, detalhando cada um de seus componentes em artigos distintos \cite{Calheiros:2012:CSE:2263483.2264538}\cite{Garg2013} \cite{4539666} \cite{6063003}. Entretanto, o sistema descrito não menciona nada sobre virtualização de máquinas compostas por arquiteturas heterogêneas.

Chen\cite{7835207} explora as vantagens de uso de nuvens horizontais e verticais concomitantemente, utilizando a teoria dos jogos para auxiliar no processo de decisão sobre quando fazer \textit{outsoucing} de \textit{workload}, explorando verticalidade da federação e quando remanejar o \textit{workload} para instâncias da coalisão(nuvem horizontal). Um dos resultados obtidos revela que o é melhor para a performance de uma federação de nuvens explorar tanto verticalmente quanto horizontalmente as nuvens existentes.

Margheri \cite{FaaS_8030651} proprõe uma arquitetura de federações de nuvens completamente distribuída, que utiliza uma \textit{blockchain} para prover um controle democrático da federação. Com grande foco em segurança e estabilidade, a arquitetura proposta provê a cada um dos membros os mesmo nível de controle sobre a federação, utiliza criptografia para tornar anônimo a origem dos serviços de computação. E utiliza \textit{blockchain} para prover democracia na gestão da federação.

Alansari\cite{ACS_Federation_7980160}, propõe um sistema de controle de acesso com alta granularidade com o objetivo de provê compartilhamento seguro de informações, com grande foco na segurança e privacidade dos usuários, especificando um protocolo de troca de chaves para processamento remoto de dados na nuvem. Também fazendo uso de \textit{blockchain} para provaer integridade de chaves públicas e outras informações necessárias do protocolo. Esse artigo faz uso de tecnologia proprietária em \acrshort{CPU}s da Intel para fazer o processamento, o que significa que não possui nenhum porte para uso em equipamentos compostos por arquiteturas heterogêneas.

Gallico\cite{CYCLONE_7776591} apresenta a federação de nuvens CYCLONE, uma federação de nuvens em desenvolvimento com o objetivo tanto comercial quanto científico, que utiliza as ferramentas \textit{open-source} \textit{OpenNaaS}, \textit{OpenStack}, \textit{SplitStream} e \acrfull{TCTP}.

Jrad\cite{Jrad:2013:BFM:2462326.2462339} propõe um \textit{framework} para executar \textit{workflows} baseado no uso de um mediador entre os os usuários e os provedores das nuvens. O mediador é capaz de negociar \acrshort{SLA}s para buscar provedores que consigam garantir entrega do serviço a um custo menor. Faz uso de um \textit{workflow engine} para processamento e clusterização de \textit{workflows} requisitados. Resultados obtidos experimentalmente mostram redução significativa no custo do processamento dos \textit{workflows}.

Mashayekhy\cite{6853386} proprõe o uso da teoria dos jogos para provedores de nuvens se aliarem em federações para situações em que sua infraestrutura não é capaz de lidar com a demanda de serviços. A teoria dos jogos é utilizada com o objetivo de determinar as melhores nuvens externas para \textit{outsourcing} do excesso de demanada.

Demchenko\cite{6427607} apresenta resultados obtidos em uma pesquisa pesquisa em andamento sobre \textit{framework} para provisionamento de infraestrutua de nuvem \textit{on-demand}. O artigo complementa a arquitetura da nuvem computacional de forma a simplificar federação horizontal para os tipos principais de serviços de nuvem(\acrshort{IaaS}, \acrshort{PaaS} e \acrshort{SaaS}).

Marosi\cite{FCM} propõe uma arquitetura de camadas que interage com intermediários de intermediários de outras nuvens para interação com o máximo possível de provedores para prover infraestrutura sob demanda tanto para nuvens públicas quanto nuvens privadas.

Zant\cite{6814036} explora federações horizontais de nuvem \acrfull{CSP} \textit{side} com objetivo de maximinizar os lucros das mesmas. Essas federações são invisíveis ao cliente da nuvem e considera que os \acrshort{CSP} participantes da federação possuem equipamentos semelhantes. Os testes provam aumento no faturamento, entretanto, \acrshort{GPU}s não são citadas ou utilizadas nos testes.

%Jennings\cite{Jennings:2015:RMC:2793474.2793493} fez o \textit{survey} mais completo sobre nuvens computacionais que o autor tem conhecimento. 


O capítulo seguinte apresenta os escalonares, seus tipos e como eles existem no contexto de federação de nuvens computacionais.

