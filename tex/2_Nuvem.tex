\section{Definição de Computação em Nuvem}

Por alguns anos a computação em nuvem não possuía uma definição bem definida principalmente por ser uma tecnologia nova. Muitas vezes sendo confundida com computação em \textit{Grid}. Atualmente sua definição já está bem definida, com duas principais definições\cite{NIST_CLOUD_DEFINITION}\cite{Vaquero:2008:BCT:1496091.1496100_Cloud_definition}, que explicitam as seguintes características de uma nuvem computacional: 

\begin{itemize}
	\item \textit{On-demand self-service};
	\item Fácil acesso via rede;
	\item Recursos virtualizados;
	\item Alta escalabilidade, com capacidade de redimensionamento durante a execução; e,
	\item Serviço mensurado.
\end{itemize}

Analisando o conjunto de caracterísitcas supracitado é possível visualizar as vantagens dessa tecnologia: reduz o investimento inicial de empresas em infraestrutura de TI; a escalabilidade também remove o custo de reinvestimento na infraestrutura, além de não se desperdiça investimentos feitos em servidores durantes vales de uso, se paga por aquilo que se consome. Entretanto nem tudo são flores, um dos maiores desafios da computação em nuvem é segurança e privacidade, pois é necessário enviar dados da empresa para a nuvem, que pode não estar sob controle da empresa. Mas há tipos de nuvem que apresentam menos riscos em termos de segurança e privacidade.

\section{Tipos de Nuvens}

A computação em nuvem é uma evolução natural da computação em \textit{Grid}, no qual a principal característica que o distingue da computação em nuvem é que na última os recursos computacionais são virtualizados, além do fácil acesso via Internet. Na computação em nuvem, existem dois atores principais: os usuários de um serviço que está na Nuvem e o provedor da Nuvem. Existem três tipos de serviços pricipais que são\cite{NIST_CLOUD_DEFINITION}:

\begin{itemize}
	\item \textbf{\textit{Infrastructure As A Service:}} Também chamado de \textit{Metal As A Service:}, o provedor disponibiliza recursos computacionais virtualizados diretamente, em termos de máquinas virtuais. Capazes de serem reconfigurados dinamicamente, pelo serviço de elasticidade;
	\item \textbf{\textit{Plataform As A Service:}} Uma plataforma sobre a qual os usuários podem implantar softwares ou serviços é disponibilizada. Os programas implanatados podem fazer uso de algumas funcionalidade fornecidas em forma de bibliotecas, linguagens de programação, entre outros. Um exemplo básico é o serviço de hospedagem de sites, que são disponibilizados no \acrshort{AWS} e no \acrshort{GCP};
	\item \textbf{\textit{Software As A Service:}} O usuário tem acesso a softwares que estão sendo executados na Nuvem Computacional. O que permite migrar requisito computacional do equipamento onde o serviço executa para a Nuvem. Um exemplo bastante conhecido é o \textit{Google Docs}\cite{GoogleDocs}.
\end{itemize}

Anteriormente categorizou-se nuvens computacionais pelos serviços providos pelas mesmas, mas essa não é a única categorização possível. Elas também são categorizadas de acordo com a forma em que são implantadas. Podendo ser públicas, privadas, híbridas e comunitárias, descritas abaixo\cite{NIST_CLOUD_DEFINITION}:

\begin{itemize}
	\item \textbf{Nuvem pública}: Criada para uso pelo público em geral.
	\item \textbf{Nuvem privada}: Quando a infraestrutura da nuvem é provida por uma única organização, para uso interno. Podendo ser gerenciada e/ou operada por um terceiro. Nuvens privadas são uma solução para o problema de privacidade dos dados. Pois esta pode estar disponível apenas internamente na organização.
	\item \textbf{Nuvem comunitária}: Nuvem com infraestrutura desenvolvida com o objetivo de ser utilizada por um grupo de pessoas e/ou organizações para fins comuns. Podendo de gerenciar por uma ou mais partes dos interessadas na nuvem.
	\item \textbf{Nuvem híbrida}: Combinação de duas ou mais das opções anteriores. Geralmente unificadas através de padronizações protocolos de comunicação ou uso de tecnologia proprietária.
\end{itemize}

[falar mais um pocuo sobre nuvens]

\section{Federações de Nuvens}

Como cada provedor de nuvem possui seu próprio método de precificação, e uma interface própria para comunicação com os usuários de seus serviços, existe um movimento de federação de nuvens computacionais com o objetivo de minimizar dependência para com os provedores do serviço e também reduzir custos.

