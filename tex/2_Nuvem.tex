\section{Definição de Computação em Nuvem}

Por alguns anos a computação em nuvem não possuía uma definição bem definida principalmente por ser uma tecnologia nova. Muitas vezes sendo confundida com computação em \textit{Grid}. Atualmente sua definição já está bem definida, com duas principais definições\cite{NIST_CLOUD_DEFINITION}\cite{Vaquero:2008:BCT:1496091.1496100_Cloud_definition}, que explicitam as seguintes características de uma numvem computacional: 

\begin{itemize}
	\item \textit{On-demand self-service};
	\item Fácil acesso via rede;
	\item Recursos virtualizados.
	\item Alta escalabilidade, com capacidade de redimensionamento durante a execução; e,
	\item Serviço mensurado.
\end{itemize}

Computação em nuvem é considerado um novo paradigma para provisionamento de infraestrutura de computação, o qual transfere o local da infraestrutura para a rede. Porém parte dos conceitos sobre os quais esse conceito se baseia não são novos.\cite{Vaquero:2008:BCT:1496091.1496100_Cloud_definition}CITAR CITAÇÕES DO ARTIGO BASE.
Na Computação em Nuvem, existem dois atores principais: os usuários de um serviço que está na Nuvem e o provedor da Nuvem. Existem três tipos de serviços pricipais que são\cite{NIST_CLOUD_DEFINITION}:

\begin{itemize}
	\item \textbf{\textit{Infrastructure As A Service:}} O provedor disponibiliza recursos computacionais virtualizados. Capazes de serem reconfigurados dinamicamente;
	\item \textbf{\textit{Plataform As A Service:}} Uma plataforma sobre a qual os usuários podem implantar softwares ou serviços é disponibilizada. Os programas implanatado podem fazer uso de algumas funcionalidade fornecidas em forma de bibliotecas, linguagens de programação, entre outros;
	\item \textbf{\textit{Software As A Service:}} O usuário tem acesso a softwares que estão sendo executados na Nuvem Computacional. O que permite migrar requisito computacional do equipamento onde o serviço executa para a Nuvem.
\end{itemize}

Como cada provedor de nuvem possui seu próprio método de precificação, e uma interface própria para comunicação com os usuários de seus serviços, existe um movimento de federação de nuvens computacionais com o objetivo de minimizar dependência para com os provedores do serviço e também reduzir custos.
