Neste trabalho foi proposto um algoritmo de escalonamento para federações de nuvens compostas por arquiteturas heterogêneas, com o objetivo de tirar proveito de placas de processamento gráfico, que possuem grande capacidade de processamento paralelo. Para tal, foi proposto um algoritmo de escalonamento baseado em listas, que leva em consideração o tempo estimado de execução da tarefa, o benefício de rodar a tarefa em \acrshort{GPU}, e a capacidade de processamento tanto da \acrshort{CPU} quanto da \acrshort{GPU}.

O escalonador proposto foi implementado em C++, e integrado, utilizando \textit{sockets} \acrshort{IP}v6, à plataforma de federação de nuvens computacionais BioNimbuZ, desenvolvida em Java. Para a integração foi desenvolvido um mecanismo de inicialização e \textit{handshake} entre as partes C++ e Java do novo escalonador, junto com um sistema de comunicação entre essas partes, que simplifica integração de futuros escalonadores em C++ ao BioNimbuZ. Além disso, também foi desenvolvido um esquema de serialização e desserialização de requisições de escalonamento via rede.

Para testes da plataforma o software de mineração de criptomoedas XMR-Stak foi adaptado e integrado ao BioNimbuZ, aumentando o catálogo de tarefas disponíveis para execução na plataforma. Entretanto, problemas na migração da máquina virtual no qual todo o ecossistema do Escalonador com o BioNimbuZ estava integrado impediram que a plataforma fosse testada integralmente, por outro lado, testes de desempenho dos escalonador desenvolvido mostra que sua performance é equivalente a do escalonador que estava em uso anteriormente, e testes preliminares do \textit{software} de XMR-Stak fora da plataforma mostram ganho de desempenho em torno de 55\%, confirmando a relevância do escalonador desenvolvido.


Como trabalho futuro, após terminar integração do escalonador com o BioNimbuZ, a metodologia de previsão de duração das tarefas pode ser refinada, através do uso de regressão linear, entre outras técnicas. Se o BioNimbuZ obtiver suporte a execução de uma mesma tarefa de forma distribuída, o algoritmo desenvolvido pode ser adaptado para dar suporte a essa nova funcionalidade. Além disso, o processo de comunicação por \textit{sockets} pode ser otimizado, pois atualmente as requisições são transmitidas em forma de texto legível, para facilitar a depuração da comunicação. Como também pode-se buscar evoluir os demais escalonadores existentes no BioNimbuZ para que eles possam tratar o caso das nuvens compostas por arquiteturas heterogêneas, além de adição de novas tarefas ao BioNimbuZ que explorem essa tecnologia.

%Diante do exposto, 
%Finalmente
